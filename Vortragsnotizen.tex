\documentclass[11pt]{scrartcl}
\usepackage[T1]{fontenc}
\usepackage[a5paper, left=3cm, right=2cm, top=2cm, bottom=2cm]{geometry}
\usepackage[activate]{pdfcprot}
\usepackage[ngerman]{babel}
\usepackage[parfill]{parskip}
\usepackage[utf8]{inputenc}
\usepackage{kurier}
\usepackage{amsmath}
\usepackage{amssymb}
\usepackage{xcolor}
\usepackage{epstopdf}
\usepackage{txfonts}
\usepackage{fancyhdr}
\usepackage{graphicx}
\usepackage{prettyref}
\usepackage{hyperref}
\usepackage{eurosym}
\usepackage{setspace}
\usepackage{units}
\usepackage{eso-pic,graphicx}
\usepackage{icomma}

\definecolor{darkblue}{rgb}{0,0,.5}
\hypersetup{pdftex=true, colorlinks=true, breaklinks=false, linkcolor=black, menucolor=black, pagecolor=black, urlcolor=darkblue}



\setlength{\columnsep}{2cm}


\newcommand{\arcsinh}{\mathrm{arcsinh}}
\newcommand{\asinh}{\mathrm{arcsinh}}
\newcommand{\ergebnis}{\textcolor{red}{\mathrm{Ergebnis}}}
\newcommand{\fehlt}{\textcolor{red}{Hier fehlen noch Inhalte.}}
\newcommand{\betanotice}{\textcolor{red}{Diese Aufgaben sind noch nicht in der Übung kontrolliert worden. Es sind lediglich meine Überlegungen und Lösungsansätze zu den Aufgaben. Es können Fehler enthalten sein!!! Das Dokument wird fortwährend aktualisiert und erst wenn das \textcolor{black}{beta} aus dem Dateinamen verschwindet ist es endgültig.}}
\newcommand{\half}{\frac{1}{2}}
\renewcommand{\d}{\, \mathrm d}
\newcommand{\punkte}{\textcolor{white}{xxxxx}}
\newcommand{\p}{\, \partial}
\newcommand{\dd}[1]{\item[#1] \hfill \\}

\renewcommand{\familydefault}{\sfdefault}
\renewcommand\thesection{}
\renewcommand\thesubsection{}
\renewcommand\thesubsubsection{}


\newcommand{\themodul}{Lasertechnik}
\newcommand{\thetutor}{Prof. Rateike}
\newcommand{\theuebung}{Übung 3}

%\pagestyle{fancy}
%\fancyhead[L]{\footnotesize{C. Hansen}}
%\chead{\thepage}
%\rhead{}
%\lfoot{}
%\cfoot{}
%\rfoot{}

\title{\themodul{}, \theuebung{}, \thetutor}


\author{Christoph Hansen \\ {\small \href{mailto:uni@christophhansen.eu}{uni@christophhansen.eu}} }

\date{}


\begin{document}

Dieser Text ist unter der
\href{http://creativecommons.org/licenses/by-nc/4.0/}{Creative Commons CC BY-NC 4.0}
Lizenz veröffentlicht.

\textcolor{red}{%
    Ich erhebe keinen Anspruch auf Vollständigkeit oder Richtigkeit. Falls ihr
    Fehler findet oder etwas fehlt, dann meldet euch bitte über den
    Emailkontakt.
}

\hfill\\

\href{mailto:chris@university-material.de}{chris@university-material.de}


\section{Folie 3}


\begin{itemize}
\item Orientierung der Galaxie gegenüber dem Beobachter
\end{itemize}


\section{Folie 4+5}

\begin{itemize}
\item erste Idee von Hubbleselber war Entwicklungssequenz von Galaxien, später negiert
\item zu geringe Anzahl - wenig hundert
\item guckte nur im optisch sichtbaren Spektrum
\item zuerst nur drei Typen - später weitere Untertypen wie Zwerggalaxien, AGNs, irreguläre Galaxien
\item nutzlos für Leuchtkraft und Entfernungsmessung - andere Verfahren müssen eingesetzt werden
\end{itemize}

\newpage


\section{Folie 6}

links oben + unten

\begin{itemize}
\item wenig blau - eher rot --> bedeutet alte Sterne
\item Vergleich  Ellipse mit Post - Starbust --> Ähnlichkeiten im Spektrum wenn auch weniger hell (ausgebrannt)
\end{itemize}


recht oben + unten

\begin{itemize}
\item viel heller als andere Galaxien --> viele junge schwere Sterne
\item 3 signifikante peaks kommen von drei Arten Sterne
\subitem > jung, schwer, kurzlebig (470- 400)
\subitem > Sonnenähnliche Sterne (um 500)
\subitem > langlebige, alte Sterne (um 660)
\end{itemize}


\section{Folie 7+8}


Emission:

\begin{itemize}
\item Analyse von Röntgen bis Radio $\unit[1]{pm} - \unit[100]{m}$ --> Teleskop Efelsberg
\item verschiedne Wellenlänge - verschiedene Schichten in der Galaxie
\end{itemize}

Absorbtion:

\begin{itemize}
\item indirekter Nachweis - Prinzip Linse erläutern
\item Information über Zusammensetzung - Was wird an Strahlung durchgelassen und was nicht
\end{itemize}



\section{Folie 9}

\begin{itemize}
\item elliptsche Isophoten (linien gleicher Dichte) ohne Spiralarme
\item große Halbachse a, kleine halbachse b 
\item Notation ist nur für schöne Zahlen da
\end{itemize}


\section{Folie 10}

\begin{itemize}
\item Erklärung Kurve + Parameter
\item dient nur der weiteren Einteilung - keine physikalische Bedeutung
\end{itemize}


\section{Folie 11}

\begin{itemize}
\item breite Gruppe
\subitem > Normale Ellpisen, cD Ellipsen, Zwerg Ellipsen, kompakte blaue Zwerge (viel Gas, blau wegen Sternentstehung)
\item cD Galaxien
\subitem > Radius $\approx$ 1 Mpc 
\subitem > Flächenhelligkeit im Zentrum sehr groß 
\subitem > ausgedehnte diffuse Hülle
\subitem > M/L Verhältnis groß 
\subitem > sind häufig in der Nähe des Zentrum eine Galaxienhaufens
\end{itemize}



\section{Folie 12}

\begin{itemize}
\item De-Vaucouleurs-Profil beschreibt die Oberflächenhelligkeit einer elliptischen Galaxie
\end{itemize}

\begin{align*}
&I(R) = \text{Flächenhelligkeit} \\
&I_e = \text{Flächenhelligkeit am Effektivradius} \\
&R_e = \text{Halblichtradius / Radius in der die Hälfte der Leuchtkraft emitiert wird}
\end{align*}

\hfill \\

NGC = New General Catalogue, Nummer = Nummer im Katalog

\hfill \\

\begin{itemize}
\item Gesamtleuchtkraft ist sowas wie \textit{Wie viele Photonen kommen von der ganzen Galaxie hier an}
\item Profile passen am besten für normale Ellipsen
\subitem > große Leuchtkraft --> sanfter Abfall
\subitem > kleine Leuchtkraft ---> schnellerer Abfall
\subitem > Zwergellipsen haben ein eher exponentielles Profil
\end{itemize}


\section{Folie 13}


\begin{itemize}
\item linearer Verlauf der Helligkeiten 
\end{itemize}


\newpage

\section{Folie 14}

\begin{itemize}
\item Ellipsen sind rot --> alte Sternpopulation
\item enthalten wenig Gas -> Gas schon von den Sternen verbraucht
\item Heißes Gas --> Röntgenemission diffus verteilt in einhüllender Korona (aüßere Halo)
\item $H_\alpha$ bei $\unit[656,281]{nm}$ -->  Hauptquantenzahlen $n=3$, $m=2$ --> Balmer Serie --> in Sternumgebung
\item kaltes Gas: Radiobereich $\unit[21]{cm}$ --> kommt vom Spinflip beim Wasserstoff --> Bestimmung der Verteilung des Wasserstoffs in der Galaxie
\item Metall ungleiche Verteilung --> steig von außen nach innen --> sterne innen älter
\item Staub: ca $\unit[50]{\%}$ der Ellipsen enthalten Staub --> entweder als Scheibe oder auf andere Art
\end{itemize}


\section{Folie 15}

\begin{itemize}
\item Sternbewegung: chaotisch --> Stöße?
\item Stöße: galaxien sind leer --> wenige Aufeinandertreffen, Gravitationsfeld der Galaxiezentrums dominiert
\item Rotationsgeschwindikeit berechenbar aus Abplattung des Ellipse und Geschwindigkeitsdispersion
\end{itemize}



\section{Folie 16}

\begin{itemize}
\item oben: Öffnungswinkel wir immer größer $\unit[6-18]{^\circ}$ --> Helligkeit den Zentrum im Vergleich zu Scheibe sinkt
\item Balken wird immer größer 
\item alles wächst von a nach d
\end{itemize}



\section{Folie 17}

\begin{itemize}
\item zwei Strukturen Bulge und Scheibe --> unterschiedliche Beschreibung der Helligkeit!
\item $R_e$ Effektivradius des Bulge, $\mu_e$ Flächenhelligkeit bei Effektivradius
\item $\mu_0$ Helligkeit der Scheibe im Zentrum (muss approximiert werden), $r_0$ Radius der Galaxie $\approx$ typisch $\unit[100000]{Lj}$
\end{itemize}


\newpage

\section{Folie 18 + 19}

\begin{itemize}
\item mitte bei $\unit[1700]{km/s}$ --> geschwindigkeit steig nach links an --> rotiert gegen den Uhrzeigersinn
\item Rotationsgeschwindigkeit bleibt bis zum Ende konstant --> ist bei alle Spiralgalaxien so
\item je höher die Leuchtkraft, desto größer die maximale Rotationsgeschwindigkeit
\item in den Außenbezirken zu wenig sichtbare Materie , die das erklären könnte
\item Lösung --> dunkle Materie, die sich bevorzugt in den Außenbereichen sammelt --> Anteil berechenbar mit Formel
\end{itemize}


\section{Folie 20}

\begin{itemize}
\item Sternentstehung fast nur in Spiralgalaxien
\item je jünger die Galaxie ist, desto mehr Sterne entstehen, da mehr Gas vorhanden ist
\item Sterne entstehen auf Grund der gravitativen Struktur der Arme fast ausschließlich dort --> in der Mitte röter, da Sterne älter --> höhere Metallizität --> geringerer Gasgehalt
\end{itemize}


\newpage

\section{Folie 21}

\begin{itemize}
\item Dichte in den Armen um $\unit[10-20]{\%}$ größer als in der Umgebung --> Gas wird bei Eintritt komprimiert
\item komprimiertes Gas zwingend für Sternentstehung nötig
\item Sterne verlassen die Arme nicht sichtbar --> heißt brennen aus, bevor die Dichtewelle an ihnen vorbei ist
\end{itemize}



\end{document}